\documentclass[10pt,a4paper]{article}

\input{AEDmacros}
\usepackage{caratula} % Version modificada para usar las macros de algo1 de ~> https://github.com/bcardiff/dc-tex

\titulo{Trabajo práctico 1}
\subtitulo{Especificación y WP}

\fecha{\today}

\materia{Algoritmos y Estructura de Datos}
\grupo{SinGrupo4}

\integrante{Algañaraz, Franco}{001/01}{francoarga10@gmail.com}
\integrante{Illescas, Marcos}{390/14}{marcosillescas90@gmail.com}
\integrante{Bahamonde, Matias}{003/01}{matubaham@gmail.com}
\integrante {Marión, Ian Pablo}{004/01}{ianfrodin@gmail.com}
% Pongan cuantos integrantes quieran

% Declaramos donde van a estar las figuras
% No es obligatorio, pero suele ser comodo
\graphicspath{{../static/}}

\begin{document}

\maketitle

\section{Especifiación}


\subsection{grandesCiudades}

\begin{proc}{grandesCiudades}{\In ciudades :\TLista{struct<nombre: string, habitantes: \ent>}}{\TLista{struct<nombre: string, habitantes: \ent>}}
	\requiere{\True}
	\asegura{\longitud{res} = (\sum_{i=0}^{|ciudades|-1} \IfThenElse{ciudades[i].habitantes > 50000}{1}{0}) \yLuego 
          (\forall i: Z)(0 \leq i < \longitud{res} \rightarrow (\exists j: Z)(0 \leq j < \longitud{ciudades} \yLuego res[i] = ciudades[j] \yLuego ciudades[j].habitantes > 50000))}
\end{proc}

\subsection{sumaDeHabitantes}

\begin{proc}{sumaDeHabitantes}{\In menoresDeCiudades : \TLista{struct<nombre: string, habitantes: \ent>}, \In mayoresDeCiudades : \TLista{struct<nombre: string, habitantes: \ent>}}{\TLista{struct<nombre: string, habitantes: \ent>}}
	\requiere{|menoresDeCiudades| = |mayoresDeCiudades| \yLuego 
          (\forall i: Z)(0 \leq i < |menoresDeCiudades| \rightarrow \existe{j}{\ent}{menoresDeCiudades[i].nombre = mayoresDeCiudades[j].nombre})}
	\asegura{|res| = |menoresDeCiudades| \yLuego 
          (\forall i: Z)(0 \leq i < |res| \rightarrow (\exists j,k : \ent)(0 \leq j < |menoresDeCiudad| \land 0 \leq k < |mayoresDeCiudad|\yLuego res[i].nombre = menoresDeCiudad[j].nombre = mayoresDeCiudad[k].nombre \implicaLuego res[i].habitantes = menoresDeCiudad[j].habitantes + mayoresDeCiudad[k].habitantes))}
\end{proc}

\subsection{hayCamino}

\begin{proc}{hayCamino}{\In distancias : \TLista{\TLista{\ent}}, \In desde : \ent, \In hasta : \ent}{\bool}
	\requiere{0 \leq desde < |distancias| \land 0 \leq hasta < |distancias|}
	\asegura{res = \True \iff existeCamino(distancia, desde, hasta)}
\end{proc}

\pred{existeCamino}{\In distancias : \TLista{\TLista{\ent}}, \In desde : \ent, \In hasta : \ent}{\existe[unalinea]{camino}{\TLista{\ent}}{|camino| \geq 2 \land camino[0] = desde \land camino[|camino| - 1] = hasta \land 
          (\forall i: \ent)(0 \leq i < |camino| - 1 \rightarrow distancias[camino[i]][camino[i+1]] > 0)}}

\subsection{cantidadCaminosNSaltos}

\begin{proc}{cantidadCaminosNSaltos}{\In conexion : \TLista{\TLista{\ent}}, \In n : \ent}{\TLista{\ent}}
	\requiere{n \geq 1 \land conexion = C_0}
	\asegura{conexion = {C_0}^n}
\end{proc}

\subsection{caminoMínimo}

\begin{proc}{caminoMinimo}{\In origen : \nat, \In destino : \ent, \In distancias : \TLista{\TLista{\ent}}}{\TLista{\ent}}
	\requiere{0 \leq origen < |distancias| \land 0 \leq destino < |distancias|}
	\asegura{(res = \lvacia \iff ¬existeCamino(distancias, origen, destino) \lor origen = destino) \land (res \neq \lvacia \implies (\forall camino : \TLista{\ent})(camino[0] = origen \land camino[|camino| - 1] = destino \land sumaDistancias(res, distancias) \leq sumaDistancias(camino, distancias)))}
    \aux{sumaDistancia}{\In camino : \TLista{\ent}, \In distancias : \TLista{\TLista{\ent}}}{\ent}{\sum_{i=0}^{|camino|-2} distancias[camino[i]][camino[i+1]]}
\end{proc}



\section{Demostraciones de correctitud}

\subsection{Demostrar que la implementación es correcta con respecto a la especificación.}

\subsection{Demostrar que el valor devuelto es mayor a 50.000.}

\end{document}