\documentclass[10pt,a4paper]{article}

\usepackage[spanish,activeacute,es-tabla]{babel}
\usepackage[utf8]{inputenc}
\usepackage{ifthen}
\usepackage{listings}
\usepackage{dsfont}
\usepackage{subcaption}
\usepackage{amsmath}
\usepackage[strict]{changepage}
\usepackage[top=1cm,bottom=2cm,left=1cm,right=1cm]{geometry}%
\usepackage{color}%
\newcommand{\tocarEspacios}{%
	\addtolength{\leftskip}{3em}%
	\setlength{\parindent}{0em}%
}

% Especificacion de procs

\newcommand{\In}{\textsf{in }}
\newcommand{\Out}{\textsf{out }}
\newcommand{\Inout}{\textsf{inout }}

\newcommand{\encabezadoDeProc}[4]{%
	% Ponemos la palabrita problema en tt
	%  \noindent%
	{\normalfont\bfseries\ttfamily proc}%
	% Ponemos el nombre del problema
	\ %
	{\normalfont\ttfamily #2}%
	\
	% Ponemos los parametros
	(#3)%
	\ifthenelse{\equal{#4}{}}{}{%
		% Por ultimo, va el tipo del resultado
		\ : #4}
}

\newenvironment{proc}[4][res]{%
	
	% El parametro 1 (opcional) es el nombre del resultado
	% El parametro 2 es el nombre del problema
	% El parametro 3 son los parametros
	% El parametro 4 es el tipo del resultado
	% Preambulo del ambiente problema
	% Tenemos que definir los comandos requiere, asegura, modifica y aux
	\newcommand{\requiere}[2][]{%
		{\normalfont\bfseries\ttfamily requiere}%
		\ifthenelse{\equal{##1}{}}{}{\ {\normalfont\ttfamily ##1} :}\ %
		\{\ensuremath{##2}\}%
		{\normalfont\bfseries\,\par}%
	}
	\newcommand{\asegura}[2][]{%
		{\normalfont\bfseries\ttfamily asegura}%
		\ifthenelse{\equal{##1}{}}{}{\ {\normalfont\ttfamily ##1} :}\
		\{\ensuremath{##2}\}%
		{\normalfont\bfseries\,\par}%
	}
	\renewcommand{\aux}[4]{%
		{\normalfont\bfseries\ttfamily aux\ }%
		{\normalfont\ttfamily ##1}%
		\ifthenelse{\equal{##2}{}}{}{\ (##2)}\ : ##3\, = \ensuremath{##4}%
		{\normalfont\bfseries\,;\par}%
	}
	\renewcommand{\pred}[3]{%
		{\normalfont\bfseries\ttfamily pred }%
		{\normalfont\ttfamily ##1}%
		\ifthenelse{\equal{##2}{}}{}{\ (##2) }%
		\{%
		\begin{adjustwidth}{+5em}{}
			\ensuremath{##3}
		\end{adjustwidth}
		\}%
		{\normalfont\bfseries\,\par}%
	}
	
	\newcommand{\res}{#1}
	\vspace{1ex}
	\noindent
	\encabezadoDeProc{#1}{#2}{#3}{#4}
	% Abrimos la llave
	\par%
	\tocarEspacios
}
{
	% Cerramos la llave
	\vspace{1ex}
}

\newcommand{\aux}[4]{%
	{\normalfont\bfseries\ttfamily\noindent aux\ }%
	{\normalfont\ttfamily #1}%
	\ifthenelse{\equal{#2}{}}{}{\ (#2)}\ : #3\, = \ensuremath{#4}%
	{\normalfont\bfseries\,;\par}%
}

\newcommand{\pred}[3]{%
	{\normalfont\bfseries\ttfamily\noindent pred }%
	{\normalfont\ttfamily #1}%
	\ifthenelse{\equal{#2}{}}{}{\ (#2) }%
	\{%
	\begin{adjustwidth}{+2em}{}
		\ensuremath{#3}
	\end{adjustwidth}
	\}%
	{\normalfont\bfseries\,\par}%
}

% Tipos

\newcommand{\nat}{\ensuremath{\mathds{N}}}
\newcommand{\ent}{\ensuremath{\mathds{Z}}}
\newcommand{\float}{\ensuremath{\mathds{R}}}
\newcommand{\bool}{\ensuremath{\mathsf{Bool}}}
\newcommand{\cha}{\ensuremath{\mathsf{Char}}}
\newcommand{\str}{\ensuremath{\mathsf{String}}}

% Logica

\newcommand{\True}{\ensuremath{\mathrm{true}}}
\newcommand{\False}{\ensuremath{\mathrm{false}}}
\newcommand{\Then}{\ensuremath{\rightarrow}}
\newcommand{\Iff}{\ensuremath{\leftrightarrow}}
\newcommand{\implica}{\ensuremath{\longrightarrow}}
\newcommand{\IfThenElse}[3]{\ensuremath{\mathsf{if}\ #1\ \mathsf{then}\ #2\ \mathsf{else}\ #3\ \mathsf{fi}}}
\newcommand{\yLuego}{\land _L}
\newcommand{\oLuego}{\lor _L}
\newcommand{\implicaLuego}{\implica _L}

\newcommand{\cuantificador}[5]{%
	\ensuremath{(#2 #3: #4)\ (%
		\ifthenelse{\equal{#1}{unalinea}}{
			#5
		}{
			$ % exiting math mode
			\begin{adjustwidth}{+2em}{}
				$#5$%
			\end{adjustwidth}%
			$ % entering math mode
		}
		)}
}

\newcommand{\existe}[4][]{%
	\cuantificador{#1}{\exists}{#2}{#3}{#4}
}
\newcommand{\paraTodo}[4][]{%
	\cuantificador{#1}{\forall}{#2}{#3}{#4}
}

%listas

\newcommand{\TLista}[1]{\ensuremath{seq \langle #1\rangle}}
\newcommand{\lvacia}{\ensuremath{[\ ]}}
\newcommand{\lv}{\ensuremath{[\ ]}}
\newcommand{\longitud}[1]{\ensuremath{|#1|}}
\newcommand{\cons}[1]{\ensuremath{\mathsf{addFirst}}(#1)}
\newcommand{\indice}[1]{\ensuremath{\mathsf{indice}}(#1)}
\newcommand{\conc}[1]{\ensuremath{\mathsf{concat}}(#1)}
\newcommand{\cab}[1]{\ensuremath{\mathsf{head}}(#1)}
\newcommand{\cola}[1]{\ensuremath{\mathsf{tail}}(#1)}
\newcommand{\sub}[1]{\ensuremath{\mathsf{subseq}}(#1)}
\newcommand{\en}[1]{\ensuremath{\mathsf{en}}(#1)}
\newcommand{\cuenta}[2]{\mathsf{cuenta}\ensuremath{(#1, #2)}}
\newcommand{\suma}[1]{\mathsf{suma}(#1)}
\newcommand{\twodots}{\ensuremath{\mathrm{..}}}
\newcommand{\masmas}{\ensuremath{++}}
\newcommand{\matriz}[1]{\TLista{\TLista{#1}}}
\newcommand{\seqchar}{\TLista{\cha}}

\renewcommand{\lstlistingname}{Código}
\lstset{% general command to set parameter(s)
	language=Java,
	morekeywords={endif, endwhile, skip},
	basewidth={0.47em,0.40em},
	columns=fixed, fontadjust, resetmargins, xrightmargin=5pt, xleftmargin=15pt,
	flexiblecolumns=false, tabsize=4, breaklines, breakatwhitespace=false, extendedchars=true,
	numbers=left, numberstyle=\tiny, stepnumber=1, numbersep=9pt,
	frame=l, framesep=3pt,
	captionpos=b,
}

\usepackage{caratula} % Version modificada para usar las macros de algo1 de ~> https://github.com/bcardiff/dc-tex
\usepackage{titlesec}

\setcounter{secnumdepth}{4}

\titleformat{\paragraph}
{\normalfont\normalsize\bfseries}{\theparagraph}{1em}{}
\titlespacing*{\paragraph}
{0pt}{3.25ex plus 1ex minus .2ex}{1.5ex plus .2ex}

\titulo{Trabajo práctico 1}
\subtitulo{Especificación y WP}

\fecha{\today}

\materia{Algoritmos y Estructura de Datos}
\grupo{SinGrupo4}

\integrante{Algañaraz, Franco}{1092/22}{francoarga10@gmail.com}
\integrante{Illescas, Marcos}{390/14}{marcosillescas90@gmail.com}
\integrante{Bahamonde, Matias}{694/21}{matubaham@gmail.com}
\integrante {Marión, Ian Pablo}{004/01}{ianfrodin@gmail.com}
% Pongan cuantos integrantes quieran

% Declaramos donde van a estar las figuras
% No es obligatorio, pero suele ser comodo
\graphicspath{{../static/}}

\begin{document}

\maketitle

\section{Especificación}

\pred{habitantesPositivos}{\In s: \TLista{Ciudad}}{
(\forall j: \ent)(0 \leq j < |s| \implicaLuego s[j].habitantes \geq 0)
}

\pred{noCiudadesRepetidas}{\In s: \TLista{Ciudad}}{
(\forall i: \ent)(\forall j: \ent)(0 \leq i < j < |s| \implicaLuego s[i].nombre \neq s[j].nombre)
}

\subsection{grandesCiudades}

\begin{proc}{grandesCiudades}{\In ciudades: \TLista{Ciudad}}{\TLista{Ciudad}}
	\requiere{habitantesPositivos(ciudades) \wedge noCiudadesRepetidas(ciudades)}
	\asegura{\longitud{res} = CantidadCiudadesMayor50000(ciudades) \; \wedge \\
          (\forall i: \ent)(0 \leq i < \longitud{res} \implicaLuego (\exists j: \ent)(0 \leq j < \longitud{ciudades} \yLuego (res[i] = ciudades[j] \wedge ciudades[j].habitantes > 50000)))}
\end{proc}


\aux{CantidadCiudadesMayor50000}{\In s: \TLista{Ciudad}}{\ent} \\ {$\sum_{i=0}^{|s|-1} \IfThenElse{s[i].habitantes > 50000}{1}{0}$;}

\subsection{sumaDeHabitantes}

\begin{proc}{sumaDeHabitantes}{\In menoresDeCiudades: \TLista{Ciudad}, \In mayoresDeCiudades: \TLista{Ciudad}}{\TLista{Ciudad}}
	\requiere{|menoresDeCiudades| = |mayoresDeCiudades| \; \; \wedge \; \\
          mismosNombres(menoresDeCiudades, mayoresDeCiudades) \; \; \wedge \\
          habitantesPositivos(menoresDeCiudades) \; \; \wedge \; \; habitantesPositivos(mayoresDeCiudades) \; \; \wedge \\
          noCiudadesRepetidas(menoresDeCiudades) \; \; \wedge \; \; noCiudadesRepetidas(mayoresDeCiudades)} 
	\asegura{|res| = |menoresDeCiudades| \; \wedge \\
          (\forall i: \ent)(0 \leq i < |res| \implicaLuego (\exists j: \ent)(\exists k: \ent)(0 \leq j < k < |menoresDeCiudad| \; \yLuego \\ siCoincidenNombresSumoHabitantes(res, menoresDeCiudades, mayoresDeCiudades, i, j, k)))}
\end{proc}

\pred{mismosNombres}{\In s: \TLista{Ciudad}, \In t: \TLista{Ciudad}}{(\forall i: \ent) (0 \leq i < |s| \implicaLuego (\exists j: \ent) (0 \leq j < |t| \yLuego s[i].nombre = t[j].nombre))}

\pred{siCoincidenNombresSumoHabitantes}{\In s: \TLista{Ciudad}, \In t: \TLista{Ciudad}, \In r: \TLista{Ciudad}, \In i: \ent, \In j: \ent, \In k: \ent}{(s[i].nombre = t[j].nombre \; \wedge \;  s[i].nombre = r[k].nombre) \longrightarrow  s[i].habitantes = t[j].habitantes + r[k].habitantes}


\subsection{hayCamino}

\begin{proc}{hayCamino}{\In distancias : \TLista{\TLista{\ent}}, \In desde : \ent, \In hasta : \ent}{\bool}
	\requiere{0 \leq desde < |distancias| \land 0 \leq hasta < |distancias|}
	\asegura{res = \True \iff existeCamino(distancia, desde, hasta)}
\end{proc}

\pred{existeCamino}{\In distancias : \TLista{\TLista{\ent}}, \In desde : \ent, \In hasta : \ent}{\existe[unalinea]{camino}{\TLista{\ent}}{|camino| \geq 2 \land camino[0] = desde \land camino[|camino| - 1] = hasta \land 
          (\forall i: \ent)(0 \leq i < |camino| - 1 \rightarrow distancias[camino[i]][camino[i+1]] > 0)}}

\subsection{cantidadCaminosNSaltos}

\begin{proc}{cantidadCaminosNSaltos}{\Inout conexion : \TLista{\TLista{\ent}}, \In n : \ent}
	\requiere{n \geq 1 \wedge conexion = C_0}
	\asegura{(\forall i: \ent)(0 \leq i < \longitud{C_0} \implicaLuego (\forall j: \ent)(0 \leq j < \longitud{C_0} \implicaLuego (\forall k: \ent)(0 \leq k < \longitud{C_0} \implicaLuego conexion[i][j] = \prod_{i=0}^{n -1} {C_0}[i][j]*{\sum_{i=0}^{longitud{C_0} -1} ({C_0}[i][k]*{C_0}[j][k]))) \yLuego longitud{C_0[i]} = longitud{conexion[i]})}
        \asegura{longitud{conexion} = longitud{C_0}}
\end{proc}

\subsection{caminoMínimo}

\begin{proc}{caminoMinimo}{\In origen : \nat, \In destino : \ent, \In distancias : \TLista{\TLista{\ent}}}{\TLista{\ent}}
	\requiere{0 \leq origen < |distancias| \land 0 \leq destino < |distancias|}
	\asegura{(res = \lvacia \iff ¬existeCamino(distancias, origen, destino) \lor origen = destino) \land (res \neq \lvacia \implies (\forall camino : \TLista{\ent})(camino[0] = origen \land camino[|camino| - 1] = destino \land sumaDistancias(res, distancias) \leq sumaDistancias(camino, distancias)))}
    \aux{sumaDistancia}{\In camino : \TLista{\ent}, \In distancias : \TLista{\TLista{\ent}}}{\ent}{\sum_{i=0}^{|camino|-2} distancias[camino[i]][camino[i+1]]}
\end{proc}



\section{Demostraciones de correctitud}

\subsection{Demostrar que la implementación es correcta con respecto a la especificación.}


\begin{proc}
{poblacionTotal}{\In ciudades: \TLista{Ciudad}}{\ent}
\requiere{(\exists j: \ent)(0 \leq j < |ciudades| \yLuego ciudades[j].habitantes > 50000) \wedge \\
(\forall k: \ent)(0 \leq k < |ciudades| \implicaLuego ciudades[i].habitantes \geq 0) \wedge \\
(\forall m: \ent)(\forall n: \ent)(0 \leq m < n < |ciudades| \implicaLuego ciudades[m].nombre \neq ciudades[n].nombre)
}
\asegura{res = \sum_{i=0}^{|ciudades|-1} ciudades[i].habitantes}    
\end{proc}


$ \\

S_{1}: res = 0 

S_{2}: i = 0 

while \; (i  \textless \; ciudades.length) \; do  \par \par
         S_{3}: res = res + ciudades.[i].habitantes \par
     
         S_{4}: i = i + 1 
     
endwhile    $
\\ \\

$P \equiv A \wedge B \wedge C$

$A \equiv (\exists l: \ent)(0 \leq l < |ciudades| \yLuego ciudades[l].habitantes > 50000)$ \par
$B \equiv (\forall k: \ent)(0 \leq k < |ciudades| \implicaLuego ciudades[k].habitantes \geq 0)$ \par
$C \equiv (\forall m: \ent)(\forall n: \ent)(0 \leq m < n < |ciudades| \implicaLuego ciudades[m].nombre \neq ciudades[n].nombre)$
\\ \par
$Pc \equiv res = 0 \; \wedge i = 0 \; \wedge A \wedge B \wedge C$
\\ \par

$B \equiv i < |ciudades|$
\\ \par

$I \equiv 0 \leq i \leq |ciudades| \yLuego res = \sum_{j=0}^{i-1} ciudades[j].habitantes$
\\ \par

$f_{v} \equiv |ciudades| - i$
\\ \par

Como el programa finaliza al terminar el ciclo, podemos asumir que: \par
\vspace{5px}

$Q \equivQc \equiv res = \sum_{j=0}^{|ciudades|-1} ciudades[j].habitantes$
\\

Para demostrar la correctitud del programa basta ver que son válidas las siguientes triplas de Hoare: \par
\vspace{5px}

{ \{Pc\} S_{3};S_{4} \{Qc \equiv{Q}\} \; y \; \{P\} S_{1};S_{2} \{Pc\}}
\vspace{5px}


\subsubsection{$Pc \implies wp(S_{3};S_{4},Qc)$ (Teorema del Invariante y Terminación)}

\paragraph{$Pc \implies I$}
$ Pc \equiv {res = 0 \; \wedge i = 0 \; \wedge A \wedge B \wedge C} \implies 0 \leq 0  \leq |ciudades| \yLuego 0 = \sum_{j=0}^{-1} ciudades[j].habitantes $ \equiv \par 
{\center $ 0 \leq |ciudades| \yLuego 0 = 0 \equiv$ \\
\vspace{5px}
$ True \; \yLuego \; True \equiv$ \\
\vspace{5px}
$True$ \\
\vspace{5px}
$Pc \implies I\par$}

\paragraph{$I \wedge B \implies wp(S_{3};S_{4}, I)$}
$wp(S_{3};S_{4}, I) \equiv wp(res = res + ciudades.[i].habitantes, \; wp(i = i + 1, I))$ \equiv \par
\vspace{5px}
$wp(res = res + ciudades.[i].habitantes, \; def(i+1) \yLuego I_{i+1}^{i})$ \equiv \par
\vspace{5px}
$wp(res = res + ciudades.[i].habitantes, \; True \yLuego 0 \leq i + 1 \leq |ciudades| \yLuego res = \sum_{j=0}^{i}{ciudades[j].habitantes})$ \equiv \par
\vspace{5px}
$wp(res = res + ciudades.[i].habitantes, \; \underbrace{-1 \leq i \leq |ciudades| - 1 \yLuego res = \sum_{j=0}^{i}{ciudades[j].habitantes}}_{I^{'}})$ \equiv \par
\vspace{5px}
$def(res + ciudades.[i].habitantes) \yLuego (I^{'})_{res + ciudades.[i].habitantes}^{res}$ \equiv \par
\vspace{5px}
$0 \leq i < |ciudades| \; \yLuego $ -1 \leq i \leq |ciudades| - 1 \yLuego res + ciudades.[i].habitantes = \sum_{j=0}^{i}{ciudades[j].habitantes}$ \equiv \par
\vspace{5px}
$0 \leq i < |ciudades| \yLuego res = \sum_{j=0}^{i-1}{ciudades[j].habitantes}$ \par
\vspace{20px}

$I \wedge B \equiv 0 \leq i \leq |ciudades| \yLuego res = \sum_{j=0}^{i-1} ciudades[j].habitantes \wedge i < |ciudades| \equiv$\par

\begin{center}
$0 \leq i < |ciudades| \yLuego res = \sum_{j=0}^{i-1} ciudades[j].habitantes  \implies  $ \par
\vspace{5px}
$ 0 \leq i < |ciudades| \yLuego \sum_{j=0}^{i-1} ciudades[j].habitantes = \sum_{j=0}^{i-1} ciudades[j].habitantes$ \equiv \par
\vspace{5px}
$True \yLuego True$ \equiv \par
\vspace{5px}
$True$\par
\vspace{5px}
    $I \wedge B \implies wp(S_{3};S_{4}, I)$
\end{center}

\paragraph{$I \wedge \neg B \implies Qc$}
$I \wedge \neg B \equiv 0 \leq i \leq |ciudades| \yLuego res = \sum_{j=0}^{i-1} ciudades[j].habitantes \wedge i \geq |ciudades| $ \equiv \par

\begin{center}
  $i = |ciudades| \yLuego res = \sum_{j=0}^{i-1} ciudades[j].habitantes$ \equiv \par
\vspace{5px}  
$res = \sum_{j=0}^{|ciudades|-1} ciudades[j].habitantes$ \implies \par
\vspace{5px} 
$\sum_{j=0}^{|ciudades|-1} ciudades[j].habitantes = \sum_{j=0}^{|ciudades|-1} ciudades[j].habitantes$ \equiv \par
\vspace{5px}
$True \yLuego True$ \equiv \par
\vspace{5px}
$True$\par
\vspace{5px}
$I \wedge \neg B \implies Qc$
\end{center}

\paragraph{$I \wedge  B \wedge v_{0}=f_{v} \implies wp(S_{3};S_{4}, f_{v} < v_{0})$}

$wp(S_{3};S_{4}, f_{v} < v_{0}) \equiv wp(res = res + ciudades[i].habitantes, wp(i = i + 1, |ciudades| - i < v_{0}))$ \equiv \par
\vspace{5px}
$wp(res = res + ciudades[i].habitantes, def(i+1) \yLuego (|ciudades| - i < v_{0})_{i+1}^{i}) $ \equiv \par
\vspace{5px}
$wp(res = res + ciudades[i].habitantes, True \yLuego |ciudades| - i - 1 < v_{0})$ \equiv \par
\vspace{5px}
$def(res + ciudades[i].habitante) \yLuego (|ciudades| - i - 1 < v_{0})_{res + ciudades[i].habitante}^{res}$ \equiv \par
\vspace{5px}
$0 \leq i < |ciudades| \yLuego |ciudades| - i - 1 < v_{0}$ \equiv \par
\vspace{20px}
$I \wedge  B \wedge v_{0}=f_{v} \equiv 0 \leq i \leq |ciudades| \yLuego res = \sum_{j=0}^{i-1} ciudades[j].habitantes \wedge i < |ciudades| \wedge v_{0}=|ciudades| - i$ \equiv \par
\begin{center}
    $0 \leq i < |ciudades| \yLuego res = \sum_{j=0}^{i-1} ciudades[j].habitantes \wedge v_{0}=|ciudades| - i$ \implies \par
    \vspace{5px}
    $0 \leq i < |ciudades| \yLuego |ciudades| - i - 1 < |ciudades| - i$ \equiv \par
    \vspace{5px}
    $True \yLuego True$ \equiv \par
    \vspace{5px}
    $True$ \par
    \vspace{5px}
    $I \wedge  B \wedge v_{0}=f_{v} \implies wp(S_{3};S_{4}, f_{v} < v_{0})$
\end{center}



\paragraph{$I \wedge f_{v} \leq 0 \implies \neg B$}
$I \wedge f_{v} \leq 0 \equiv 0 \leq i \leq |ciudades| \yLuego res = \sum_{j=0}^{i-1} ciudades[j].habitantes \wedge |ciudades| - i \leq 0$ \equiv \par
\begin{center}
    $0 \leq i \leq |ciudades| \yLuego res = \sum_{j=0}^{i-1} ciudades[j].habitantes \wedge |ciudades| \leq i$ \equiv \par
    \vspace{5px}
    $i = |ciudades| \yLuego res = \sum_{j=0}^{i-1} ciudades[j].habitantes $ \implies \par
    \vspace{5px}
    $i \geq |ciudades|$ \equiv \par
    \vspace{5px}
    $True$ \par
    \vspace{5px}
    $I \wedge f_{v} \leq 0 \implies \neg B$
\end{center}

\subsubsection{$P \implies wp(S_{1};S_{2}, Pc)$}
$wp(S_{1};S_{2}, Pc) \equiv wp(res = 0, wp(i = 0, Pc)) \equiv wp(res = 0, def(0) \yLuego Pc_{0}^{i})$ \equiv \par
\vspace{5px}
$wp(res = 0, True \yLuego res = 0 \wedge 0 = 0 \wedge A \wedge B \wedge C) $ \equiv \par
\vspace{5px}
$def(0) \yLuego (res = 0 \wedge True \wedge A \wedge B \wedge C)_{0}^{res} $ \equiv \par
\vspace{5px}
$True \yLuego 0 = 0 \wedge A \wedge B \wedge C \equiv True \wedge A \wedge B \wedge C \equiv  A \wedge B \wedge C $ \ \par
\vspace{20px}
$P \equiv A \wedge B \wedge C \implies A \wedge B \wedge C \equiv True \wedge True \wedge True \equiv True$ \par
\vspace{5px}
\begin{center}
    $P \implies wp(S_{1};S_{2}, Pc)$
\end{center}
El programa es correcto con respecto a su especificación.

\subsection{Demostrar que el valor devuelto es mayor a 50.000.}
Informalmente, a partir de la precondición del programa se puede asumir que existe algún elemento de la secuencia de entrada con más de 50.000 habitantes. Como el ciclo, mediante el paso de las iteraciones, suma los habitantes de cada ciudad de la entrada, no hay ciudades con un numero negativo de habitantes y el ciclo finaliza, se puede concluir que la suma total de habitantes dará como resultado al menos 50.000, cumpliendo así lo pedido. \par
Formalmente, para demostrar que el valor devuelto es mayor a 50.000, consideramos la siguiente postcondición Q y vemos si el programa sigue siendo correcto con respecto a su especificación.

Tomando $P, B, Pc, f_{v}$ igual que en el punto anterior, y \\ \par
$Q \equiv Qc \equiv res = \sum_{j=0}^{|ciudades|-1} ciudades[j].habitantes \wedge res > 50000$ \\ \par
$A \equiv (\exists l: \ent)(0 \leq l < |ciudades| \yLuego ciudades[l].habitantes > 50000)$\\ \par
$I \equiv 0 \leq i \leq |ciudades| \yLuego res = \sum_{j=0}^{i-1} ciudades[j].habitantes \wedge A$\\ \par
Para demostrar la correctitud del programa basta ver que son válidas las siguientes triplas de Hoare: \par
\vspace{5px}

\begin{center}
   ${ \{Pc\} S_{3};S_{4} \{Qc \equiv{Q}\} \; y \; \{P\} S_{1};S_{2} \{Pc\}}$
\vspace{5px} 
\end{center}


\subsubsection{$Pc \implies wp(S_{3};S_{4},Qc)$ (Teorema del Invariante y Terminación)}

\paragraph{$Pc \implies I$}
$ Pc \equiv {res = 0 \; \wedge i = 0 \; \wedge A \wedge B \wedge C} \implies 0 \leq 0  \leq |ciudades| \yLuego 0 = \sum_{j=0}^{-1} ciudades[j].habitantes \wedge A$ \equiv \par 
{\center $ 0 \leq |ciudades| \yLuego 0 = 0 \wedge True \equiv$ \\
\vspace{5px}
$ True \; \yLuego \; True \equiv$ \\
\vspace{5px}
$True$ \\
\vspace{5px}
$Pc \implies I\par$}

\paragraph{$I \wedge B \implies wp(S_{3};S_{4}, I)$}
$wp(S_{3};S_{4}, I) \equiv wp(res = res + ciudades.[i].habitantes, \; wp(i = i + 1, I))$ \equiv \par
\vspace{5px}
$wp(res = res + ciudades.[i].habitantes, \; def(i+1) \yLuego I_{i+1}^{i})$ \equiv \par
\vspace{5px}
$wp(res = res + ciudades.[i].habitantes, \; True \yLuego 0 \leq i + 1 \leq |ciudades| \yLuego res = \sum_{j=0}^{i}{ciudades[j].habitantes} \wedge A)$ \equiv \par
\vspace{5px}
$wp(res = res + ciudades.[i].habitantes, \; \underbrace{-1 \leq i \leq |ciudades| - 1 \yLuego res = \sum_{j=0}^{i}{ciudades[j].habitantes} \wedge A}_{I^{'}})$ \equiv \par
\vspace{5px}
$def(res + ciudades.[i].habitantes) \yLuego (I^{'})_{res + ciudades.[i].habitantes}^{res}$ \equiv \par
\vspace{5px}
$0 \leq i < |ciudades| \; \yLuego $ -1 \leq i \leq |ciudades| - 1 \yLuego res + ciudades.[i].habitantes = \sum_{j=0}^{i}{ciudades[j].habitantes} \wedge A$ \equiv \par
\vspace{5px}
$0 \leq i < |ciudades| \yLuego res = \sum_{j=0}^{i-1}{ciudades[j].habitantes} \wedge A$ \par
\vspace{20px}

$I \wedge B \equiv 0 \leq i \leq |ciudades| \yLuego res = \sum_{j=0}^{i-1} ciudades[j].habitantes \wedge A \wedge i < |ciudades| \equiv$\par

\begin{center}
$0 \leq i < |ciudades| \yLuego res = \sum_{j=0}^{i-1} ciudades[j].habitantes \wedge A \implies  $ \par
\vspace{5px}
$ 0 \leq i < |ciudades| \yLuego \sum_{j=0}^{i-1} ciudades[j].habitantes = \sum_{j=0}^{i-1} ciudades[j].habitantes \wedge A$  \equiv \par
\vspace{5px}
$True \yLuego True \wedge True$ \equiv \par
\vspace{5px}
$True$\par
\vspace{5px}
    $I \wedge B \implies wp(S_{3};S_{4}, I)$
\end{center}

\paragraph{$I \wedge \neg B \implies Qc$}
$I \wedge \neg B \equiv 0 \leq i \leq |ciudades| \yLuego res = \sum_{j=0}^{i-1} ciudades[j].habitantes \wedge A \wedge i \geq |ciudades| $ \equiv \par

\begin{center}
  $i = |ciudades| \yLuego res = \sum_{j=0}^{i-1} ciudades[j].habitantes \wedge A$ \equiv \par
\vspace{5px}  
$res = \sum_{j=0}^{|ciudades|-1} ciudades[j].habitantes \wedge A$ \implies \par
\vspace{5px} 
$\sum_{j=0}^{|ciudades|-1} ciudades[j].habitantes = \sum_{j=0}^{|ciudades|-1} ciudades[j].habitantes \wedge \sum_{j=0}^{|ciudades|-1} ciudades[j].habitantes > 50000$  \par
\vspace{5px}
\equiv \; $True \yLuego True$ \equiv \par
\vspace{5px}
$True$\par
\vspace{5px}
$I \wedge \neg B \implies Qc$
\end{center}

\paragraph{$I \wedge  B \wedge v_{0}=f_{v} \implies wp(S_{3};S_{4}, f_{v} < v_{0})$}

$wp(S_{3};S_{4}, f_{v} < v_{0}) \equiv wp(res = res + ciudades[i].habitantes, wp(i = i + 1, |ciudades| - i < v_{0}))$ \equiv \par
\vspace{5px}
$wp(res = res + ciudades[i].habitantes, def(i+1) \yLuego (|ciudades| - i < v_{0})_{i+1}^{i}) $ \equiv \par
\vspace{5px}
$wp(res = res + ciudades[i].habitantes, True \yLuego |ciudades| - i - 1 < v_{0})$ \equiv \par
\vspace{5px}
$def(res + ciudades[i].habitante) \yLuego (|ciudades| - i - 1 < v_{0})_{res + ciudades[i].habitante}^{res}$ \equiv \par
\vspace{5px}
$0 \leq i < |ciudades| \yLuego |ciudades| - i - 1 < v_{0}$ \equiv \par
\vspace{20px}
$I \wedge  B \wedge v_{0}=f_{v} \equiv 0 \leq i \leq |ciudades| \yLuego res = \sum_{j=0}^{i-1} ciudades[j].habitantes \wedge A \wedge i < |ciudades| \wedge v_{0}=|ciudades| - i$ \equiv \par
\begin{center}
    $0 \leq i < |ciudades| \yLuego res = \sum_{j=0}^{i-1} ciudades[j].habitantes \wedge A \wedge v_{0}=|ciudades| - i$ \implies \par
    \vspace{5px}
    $0 \leq i < |ciudades| \yLuego |ciudades| - i - 1 < |ciudades| - i$ \equiv \par
    \vspace{5px}
    $True \yLuego True$ \equiv \par
    \vspace{5px}
    $True$ \par
    \vspace{5px}
    $I \wedge  B \wedge v_{0}=f_{v} \implies wp(S_{3};S_{4}, f_{v} < v_{0})$
\end{center}



\paragraph{$I \wedge f_{v} \leq 0 \implies \neg B$}
$I \wedge f_{v} \leq 0 \equiv 0 \leq i \leq |ciudades| \yLuego res = \sum_{j=0}^{i-1} ciudades[j].habitantes \wedge A \wedge |ciudades| - i \leq 0$ \equiv \par
\begin{center}
    $0 \leq i \leq |ciudades| \yLuego res = \sum_{j=0}^{i-1} ciudades[j].habitantes \wedge A \wedge |ciudades| \leq i$ \equiv \par
    \vspace{5px}
    $i = |ciudades| \yLuego res = \sum_{j=0}^{i-1} ciudades[j].habitantes \wedge A $ \implies \par
    \vspace{5px}
    $i \geq |ciudades|$ \equiv \par
    \vspace{5px}
    $True$ \par
    \vspace{5px}
    $I \wedge f_{v} \leq 0 \implies \neg B$
\end{center}

\subsubsection{$P \implies wp(S_{1};S_{2}, Pc)$}
$wp(S_{1};S_{2}, Pc) \equiv wp(res = 0, wp(i = 0, Pc)) \equiv wp(res = 0, def(0) \yLuego Pc_{0}^{i})$ \equiv \par
\vspace{5px}
$wp(res = 0, True \yLuego res = 0 \wedge 0 = 0 \wedge A \wedge B \wedge C) $ \equiv \par
\vspace{5px}
$def(0) \yLuego (res = 0 \wedge True \wedge A \wedge B \wedge C)_{0}^{res} $ \equiv \par
\vspace{5px}
$True \yLuego 0 = 0 \wedge A \wedge B \wedge C \equiv True \wedge A \wedge B \wedge C \equiv  A \wedge B \wedge C $ \ \par
\vspace{20px}
$P \equiv A \wedge B \wedge C \implies A \wedge B \wedge C \equiv True \wedge True \wedge True \equiv True$ \par
\vspace{5px}
\begin{center}
    $P \implies wp(S_{1};S_{2}, Pc)$
\end{center}
El programa es correcto con respecto a su especificación.

\end{document}
