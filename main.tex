\documentclass[10pt,a4paper]{article}

\usepackage[spanish,activeacute,es-tabla]{babel}
\usepackage[utf8]{inputenc}
\usepackage{ifthen}
\usepackage{listings}
\usepackage{dsfont}
\usepackage{subcaption}
\usepackage{amsmath}
\usepackage[strict]{changepage}
\usepackage[top=1cm,bottom=2cm,left=1cm,right=1cm]{geometry}%
\usepackage{color}%
\newcommand{\tocarEspacios}{%
	\addtolength{\leftskip}{3em}%
	\setlength{\parindent}{0em}%
}

% Especificacion de procs

\newcommand{\In}{\textsf{in }}
\newcommand{\Out}{\textsf{out }}
\newcommand{\Inout}{\textsf{inout }}

\newcommand{\encabezadoDeProc}[4]{%
	% Ponemos la palabrita problema en tt
	%  \noindent%
	{\normalfont\bfseries\ttfamily proc}%
	% Ponemos el nombre del problema
	\ %
	{\normalfont\ttfamily #2}%
	\
	% Ponemos los parametros
	(#3)%
	\ifthenelse{\equal{#4}{}}{}{%
		% Por ultimo, va el tipo del resultado
		\ : #4}
}

\newenvironment{proc}[4][res]{%
	
	% El parametro 1 (opcional) es el nombre del resultado
	% El parametro 2 es el nombre del problema
	% El parametro 3 son los parametros
	% El parametro 4 es el tipo del resultado
	% Preambulo del ambiente problema
	% Tenemos que definir los comandos requiere, asegura, modifica y aux
	\newcommand{\requiere}[2][]{%
		{\normalfont\bfseries\ttfamily requiere}%
		\ifthenelse{\equal{##1}{}}{}{\ {\normalfont\ttfamily ##1} :}\ %
		\{\ensuremath{##2}\}%
		{\normalfont\bfseries\,\par}%
	}
	\newcommand{\asegura}[2][]{%
		{\normalfont\bfseries\ttfamily asegura}%
		\ifthenelse{\equal{##1}{}}{}{\ {\normalfont\ttfamily ##1} :}\
		\{\ensuremath{##2}\}%
		{\normalfont\bfseries\,\par}%
	}
	\renewcommand{\aux}[4]{%
		{\normalfont\bfseries\ttfamily aux\ }%
		{\normalfont\ttfamily ##1}%
		\ifthenelse{\equal{##2}{}}{}{\ (##2)}\ : ##3\, = \ensuremath{##4}%
		{\normalfont\bfseries\,;\par}%
	}
	\renewcommand{\pred}[3]{%
		{\normalfont\bfseries\ttfamily pred }%
		{\normalfont\ttfamily ##1}%
		\ifthenelse{\equal{##2}{}}{}{\ (##2) }%
		\{%
		\begin{adjustwidth}{+5em}{}
			\ensuremath{##3}
		\end{adjustwidth}
		\}%
		{\normalfont\bfseries\,\par}%
	}
	
	\newcommand{\res}{#1}
	\vspace{1ex}
	\noindent
	\encabezadoDeProc{#1}{#2}{#3}{#4}
	% Abrimos la llave
	\par%
	\tocarEspacios
}
{
	% Cerramos la llave
	\vspace{1ex}
}

\newcommand{\aux}[4]{%
	{\normalfont\bfseries\ttfamily\noindent aux\ }%
	{\normalfont\ttfamily #1}%
	\ifthenelse{\equal{#2}{}}{}{\ (#2)}\ : #3\, = \ensuremath{#4}%
	{\normalfont\bfseries\,;\par}%
}

\newcommand{\pred}[3]{%
	{\normalfont\bfseries\ttfamily\noindent pred }%
	{\normalfont\ttfamily #1}%
	\ifthenelse{\equal{#2}{}}{}{\ (#2) }%
	\{%
	\begin{adjustwidth}{+2em}{}
		\ensuremath{#3}
	\end{adjustwidth}
	\}%
	{\normalfont\bfseries\,\par}%
}

% Tipos

\newcommand{\nat}{\ensuremath{\mathds{N}}}
\newcommand{\ent}{\ensuremath{\mathds{Z}}}
\newcommand{\float}{\ensuremath{\mathds{R}}}
\newcommand{\bool}{\ensuremath{\mathsf{Bool}}}
\newcommand{\cha}{\ensuremath{\mathsf{Char}}}
\newcommand{\str}{\ensuremath{\mathsf{String}}}

% Logica

\newcommand{\True}{\ensuremath{\mathrm{true}}}
\newcommand{\False}{\ensuremath{\mathrm{false}}}
\newcommand{\Then}{\ensuremath{\rightarrow}}
\newcommand{\Iff}{\ensuremath{\leftrightarrow}}
\newcommand{\implica}{\ensuremath{\longrightarrow}}
\newcommand{\IfThenElse}[3]{\ensuremath{\mathsf{if}\ #1\ \mathsf{then}\ #2\ \mathsf{else}\ #3\ \mathsf{fi}}}
\newcommand{\yLuego}{\land _L}
\newcommand{\oLuego}{\lor _L}
\newcommand{\implicaLuego}{\implica _L}

\newcommand{\cuantificador}[5]{%
	\ensuremath{(#2 #3: #4)\ (%
		\ifthenelse{\equal{#1}{unalinea}}{
			#5
		}{
			$ % exiting math mode
			\begin{adjustwidth}{+2em}{}
				$#5$%
			\end{adjustwidth}%
			$ % entering math mode
		}
		)}
}

\newcommand{\existe}[4][]{%
	\cuantificador{#1}{\exists}{#2}{#3}{#4}
}
\newcommand{\paraTodo}[4][]{%
	\cuantificador{#1}{\forall}{#2}{#3}{#4}
}

%listas

\newcommand{\TLista}[1]{\ensuremath{seq \langle #1\rangle}}
\newcommand{\lvacia}{\ensuremath{[\ ]}}
\newcommand{\lv}{\ensuremath{[\ ]}}
\newcommand{\longitud}[1]{\ensuremath{|#1|}}
\newcommand{\cons}[1]{\ensuremath{\mathsf{addFirst}}(#1)}
\newcommand{\indice}[1]{\ensuremath{\mathsf{indice}}(#1)}
\newcommand{\conc}[1]{\ensuremath{\mathsf{concat}}(#1)}
\newcommand{\cab}[1]{\ensuremath{\mathsf{head}}(#1)}
\newcommand{\cola}[1]{\ensuremath{\mathsf{tail}}(#1)}
\newcommand{\sub}[1]{\ensuremath{\mathsf{subseq}}(#1)}
\newcommand{\en}[1]{\ensuremath{\mathsf{en}}(#1)}
\newcommand{\cuenta}[2]{\mathsf{cuenta}\ensuremath{(#1, #2)}}
\newcommand{\suma}[1]{\mathsf{suma}(#1)}
\newcommand{\twodots}{\ensuremath{\mathrm{..}}}
\newcommand{\masmas}{\ensuremath{++}}
\newcommand{\matriz}[1]{\TLista{\TLista{#1}}}
\newcommand{\seqchar}{\TLista{\cha}}

\renewcommand{\lstlistingname}{Código}
\lstset{% general command to set parameter(s)
	language=Java,
	morekeywords={endif, endwhile, skip},
	basewidth={0.47em,0.40em},
	columns=fixed, fontadjust, resetmargins, xrightmargin=5pt, xleftmargin=15pt,
	flexiblecolumns=false, tabsize=4, breaklines, breakatwhitespace=false, extendedchars=true,
	numbers=left, numberstyle=\tiny, stepnumber=1, numbersep=9pt,
	frame=l, framesep=3pt,
	captionpos=b,
}

\usepackage{caratula} % Version modificada para usar las macros de algo1 de ~> https://github.com/bcardiff/dc-tex

\titulo{Trabajo práctico 1}
\subtitulo{Especificación y WP}

\fecha{\today}

\materia{Algoritmos y Estructura de Datos}
\grupo{SinGrupo4}

\integrante{Algañaraz, Franco}{001/01}{francoarga10@gmail.com}
\integrante{Illescas, Marcos}{390/14}{marcosillescas90@gmail.com}
\integrante{Bahamonde, Matias}{003/01}{matubaham@gmail.com}
\integrante {Marión, Ian Pablo}{004/01}{ianfrodin@gmail.com}
% Pongan cuantos integrantes quieran

% Declaramos donde van a estar las figuras
% No es obligatorio, pero suele ser comodo
\graphicspath{{../static/}}

\begin{document}

\maketitle

\section{Especifiación}


\subsection{grandesCiudades}

\begin{proc}{grandesCiudades}{\In ciudades :\TLista{struct<nombre: string, habitantes: \ent>}}{\TLista{struct<nombre: string, habitantes: \ent>}}
	\requiere{\True}
	\asegura{\longitud{res} = (\sum_{i=0}^{|ciudades|-1} \IfThenElse{ciudades[i].habitantes > 50000}{1}{0}) \yLuego 
          (\forall i: Z)(0 \leq i < \longitud{res} \rightarrow (\exists j: Z)(0 \leq j < \longitud{ciudades} \yLuego res[i] = ciudades[j] \yLuego ciudades[j].habitantes > 50000))}
\end{proc}

\subsection{sumaDeHabitantes}

\begin{proc}{sumaDeHabitantes}{\In menoresDeCiudades : \TLista{struct<nombre: string, habitantes: \ent>}, \In mayoresDeCiudades : \TLista{struct<nombre: string, habitantes: \ent>}}{\TLista{struct<nombre: string, habitantes: \ent>}}
	\requiere{|menoresDeCiudades| = |mayoresDeCiudades| \yLuego 
          (\forall i: Z)(0 \leq i < |menoresDeCiudades| \rightarrow \existe{j}{\ent}{menoresDeCiudades[i].nombre = mayoresDeCiudades[j].nombre})}
	\asegura{|res| = |menoresDeCiudades| \yLuego 
          (\forall i: Z)(0 \leq i < |res| \rightarrow (\exists j,k : \ent)(0 \leq j < |menoresDeCiudad| \land 0 \leq k < |mayoresDeCiudad|\yLuego res[i].nombre = menoresDeCiudad[j].nombre = mayoresDeCiudad[k].nombre \implicaLuego res[i].habitantes = menoresDeCiudad[j].habitantes + mayoresDeCiudad[k].habitantes))}
\end{proc}

\subsection{hayCamino}

\begin{proc}{hayCamino}{\In distancias : \TLista{\TLista{\ent}}, \In desde : \ent, \In hasta : \ent}{\bool}
	\requiere{0 \leq desde < |distancias| \land 0 \leq hasta < |distancias|}
	\asegura{res = \True \iff existeCamino(distancia, desde, hasta)}
\end{proc}

\pred{existeCamino}{\In distancias : \TLista{\TLista{\ent}}, \In desde : \ent, \In hasta : \ent}{\existe[unalinea]{camino}{\TLista{\ent}}{|camino| \geq 2 \land camino[0] = desde \land camino[|camino| - 1] = hasta \land 
          (\forall i: \ent)(0 \leq i < |camino| - 1 \rightarrow distancias[camino[i]][camino[i+1]] > 0)}}

\subsection{cantidadCaminosNSaltos}

\begin{proc}{cantidadCaminosNSaltos}{\In conexion : \TLista{\TLista{\ent}}, \In n : \ent}{\TLista{\ent}}
	\requiere{n \geq 1 \land conexion = C_0}
	\asegura{conexion = {C_0}^n}
\end{proc}

\subsection{caminoMínimo}

\begin{proc}{caminoMinimo}{\In origen : \nat, \In destino : \ent, \In distancias : \TLista{\TLista{\ent}}}{\TLista{\ent}}
	\requiere{0 \leq origen < |distancias| \land 0 \leq destino < |distancias|}
	\asegura{(res = \lvacia \iff ¬existeCamino(distancias, origen, destino) \lor origen = destino) \land (res \neq \lvacia \implies (\forall camino : \TLista{\ent})(camino[0] = origen \land camino[|camino| - 1] = destino \land sumaDistancias(res, distancias) \leq sumaDistancias(camino, distancias)))}
    \aux{sumaDistancia}{\In camino : \TLista{\ent}, \In distancias : \TLista{\TLista{\ent}}}{\ent}{\sum_{i=0}^{|camino|-2} distancias[camino[i]][camino[i+1]]}
\end{proc}



\section{Demostraciones de correctitud}

\subsection{Demostrar que la implementación es correcta con respecto a la especificación.}

\subsection{Demostrar que el valor devuelto es mayor a 50.000.}

\end{document}