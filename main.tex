\documentclass[10pt,a4paper]{article}

\input{AEDmacros}
\usepackage{caratula} % Version modificada para usar las macros de algo1 de ~> https://github.com/bcardiff/dc-tex
\usepackage{titlesec}
\usepackage{array,amsmath}

\setcounter{secnumdepth}{4}

\titleformat{\paragraph}
{\normalfont\normalsize\bfseries}{\theparagraph}{1em}{}
\titlespacing*{\paragraph}
{0pt}{3.25ex plus 1ex minus .2ex}{1.5ex plus .2ex}

\titulo{Trabajo práctico 1}
\subtitulo{Especificación y WP}

\fecha{\today}

\materia{Algoritmos y Estructura de Datos}
\grupo{SinGrupo4}

\integrante{Algañaraz, Franco}{1092/22}{francoarga10@gmail.com}
\integrante{Illescas, Marcos}{390/14}{marcosillescas90@gmail.com}
\integrante{Bahamonde, Matias}{694/21}{matubaham@gmail.com}
\integrante {Marión, Ian Pablo}{004/01}{ianfrodin@gmail.com}
% Pongan cuantos integrantes quieran

% Declaramos donde van a estar las figuras
% No es obligatorio, pero suele ser comodo
\graphicspath{{../static/}}

\begin{document}

\maketitle


\section{Especificación}
\setcounter{subsection}{-1}
\subsection{Predicados recurrentes}
\pred{habitantesPositivos}{s: \TLista{Ciudad}}{
(\forall j: \ent)(0 \leq j < |s| \implicaLuego s[j].habitantes \geq 0)
}

\pred{noCiudadesRepetidas}{s: \TLista{Ciudad}}{
(\forall i: \ent)(\forall j: \ent)(0 \leq i < j < |s| \implicaLuego s[i].nombre \neq s[j].nombre)
}


\pred{matrizCuadrada}{s: \TLista{\TLista{\ent}}}{
(\forall i: \ent)(0 \leq i < |s| \implicaLuego |s| = |s[i]|)
}

\pred{matrizPositivaSimetricaConCerosDiagonal}{s: \TLista{\TLista{\ent}}}{
(\forall i: \ent)(\forall j: \ent)(0 \leq i < j < |s| \implicaLuego s[i][i] = 0 \wedge  s[i][j] = s[j][i] \wedge s[i][j] \geq 0)
}

\pred{esCamino}{camino: \TLista{\ent}, desde: \ent,  hasta: \ent, A: \TLista{\TLista{\ent}}}{
|camino| \geq 2 \yLuego camino[0] = desde \wedge camino[|camino|-1] = hasta \wedge \\(\forall i: \ent)(0 \leq i < |camino| \implicaLuego 0 \leq camino[i] < |A|) \wedge \\ (\forall j: \ent)(0 \leq j < |camino| - 1 \implicaLuego A[camino[j]][camino[j+1]] > 0)
}

\subsection{grandesCiudades}

\begin{proc}{grandesCiudades}{\In ciudades: \TLista{Ciudad}}{\TLista{Ciudad}}
	\requiere{habitantesPositivos(ciudades) \wedge noCiudadesRepetidas(ciudades)}
	\asegura{\longitud{res} = cantidadCiudadesMayor50000(ciudades) \; \wedge noCiudadesRepetidas(res) \wedge\\
          (\forall i: \ent)(0 \leq i < \longitud{res} \implicaLuego (\exists j: \ent)(0 \leq j < \longitud{ciudades} \yLuego (res[i] = ciudades[j] \wedge ciudades[j].habitantes > 50000)))}
\end{proc}


\aux{cantidadCiudadesMayor50000}{\In s: \TLista{Ciudad}}{\ent} \\ {$\sum_{i=0}^{|s|-1} \IfThenElse{s[i].habitantes > 50000}{1}{0}$;}

\subsection{sumaDeHabitantes}

\begin{proc}{sumaDeHabitantes}{\In menoresDeCiudades: \TLista{Ciudad}, \In mayoresDeCiudades: \TLista{Ciudad}}{\TLista{Ciudad}}
	\requiere{|menoresDeCiudades| = |mayoresDeCiudades| \; \; \wedge \; \\
          mismosNombres(menoresDeCiudades, mayoresDeCiudades) \; \; \wedge \\
          habitantesPositivos(menoresDeCiudades) \; \; \wedge \; \; habitantesPositivos(mayoresDeCiudades) \; \; \wedge \\
          noCiudadesRepetidas(menoresDeCiudades) \; \; \wedge \; \; noCiudadesRepetidas(mayoresDeCiudades)} 
	\asegura{|res| = |menoresDeCiudades| \; \wedge \\
          (\forall i: \ent)(0 \leq i < |res| \implicaLuego (\exists j: \ent)(\exists k: \ent)(0 \leq j < k < |menoresDeCiudad| \; \yLuego \\ siCoincidenNombresSumoHabitantes(res, menoresDeCiudades, mayoresDeCiudades, i, j, k)))}
\end{proc}

\pred{mismosNombres}{s: \TLista{Ciudad}, t: \TLista{Ciudad}}{(\forall i: \ent) (0 \leq i < |s| \implicaLuego (\exists j: \ent) (0 \leq j < |t| \yLuego s[i].nombre = t[j].nombre))}

\pred{siCoincidenNombresSumoHabitantes}{res: \TLista{Ciudad}, men: \TLista{Ciudad}, may: \TLista{Ciudad}, i: \ent,  j: \ent,  k: \ent}{(res[i].nombre = men[j].nombre \; \wedge \;  res[i].nombre = may[k].nombre) \; \wedge \\ res[i].habitantes = men[j].habitantes + may[k].habitantes}


\subsection{hayCamino}

\begin{proc}{hayCamino}{\In distancias : \TLista{\TLista{\ent}}, \In desde : \ent, \In hasta : \ent}{\bool}
	\requiere{matrizCuadrada(distancias) \wedge matrizPositivaSimetricaConCerosDiagonal(distancias) \wedge \\ 0 \leq desde < |distancias| \land 0 \leq hasta < |distancias|}
	\asegura{res = \True \longleftrightarrow (\exists c: \TLista{\ent})(esCamino(c, desde, hasta, distancias))}
\end{proc}


\subsection{cantidadCaminosNSaltos}

\begin{proc}{cantidadCaminosNSaltos}{\Inout conexion: \TLista{\TLista{\ent}}, \In n: \ent}{}
	\requiere{n \geq 1 \wedge matrizCuadrada(conexion) \wedge \\ matrizPositivaSimetricaConCerosDiagonal(conexion) \wedge unosYceros(conexion) \wedge conexion = C_0}
\asegura{(\exists s: \TLista{\TLista{\TLista{\ent}}})(|s| = n \yLuego  s[0] = C_0 \wedge \\
(\forall i: \ent)(0 \leq i < n \implicaLuego matrizCuadrada(s[i]) \wedge matrizPositivaSimetrica(s[i])) \wedge \\
(\forall j: \ent)(1 \leq j < n \implicaLuego esMultiplicacionMatricialDe(s[j], s[j-1], C_0)) \wedge conexion = s[n-1])}
\end{proc}

\pred{unosYceros}{s: \TLista{\TLista{\ent}}}{
(\forall i: \ent)(\forall j: \ent)(0 \leq i < j < |s| \implicaLuego s[i][j] = 0 \vee s[i][j] = 1)
}

\pred{matrizPositivaSimetrica}{s: \TLista{\TLista{\ent}}}{
(\forall i: \ent)(\forall j: \ent)(0 \leq i < j < |s| \implicaLuego s[i][j] = s[j][i] \wedge s[i][j] \geq 0)
}


\pred{esMultiplicacionMatricialDe}{A: \TLista{\TLista{\ent}}, B: \TLista{\TLista{\ent}}, C: \TLista{\TLista{\ent}}}{
(\forall i: \ent)(\forall j: \ent)(0 \leq i \leq j < |A| \implicaLuego A[i][j] = \sum_{k=0}^{|A|-1}{B[i][k]*C[k][j]})
}

\subsection{caminoMínimo}

\begin{proc}{caminoMinimo}{\In origen: \ent, \In destino: \ent, \In distancias: \TLista{\TLista{\ent}}}{\TLista{\ent}}
	\requiere{matrizCuadrada(distancias) \wedge matrizPositivaSimetricaConCerosDiagonal(distancias) \wedge \\ 0 \leq origen < |distancias| \land 0 \leq destino < |distancias|}
	\asegura{(|res| = 0 \wedge \neg(\exists s: \TLista{\ent})(esCamino(s, origen, hasta, distancias))) \vee \\ (esCamino(res, origen, hasta, distancias) \wedge (\forall c: \TLista{\ent})(esCamino(c, origen, destino, distancias) \longrightarrow \\
 distanciaTotal(res, distancias) \leq distanciaTotal(c, distancias)))}
    \aux{distanciaTotal}{\In s: \TLista{\ent}, \In A: \TLista{\TLista{\ent}}}{\ent}{\sum_{i=0}^{|c|-2} A[c[i]][c[i+1]]}
\end{proc}



\section{Demostraciones de correctitud}

\subsection{Demostrar que la implementación es correcta con respecto a la especificación.}


\begin{proc}
{poblacionTotal}{\In ciudades: \TLista{Ciudad}}{\ent}
\requiere{(\exists j: \ent)(0 \leq j < |ciudades| \yLuego ciudades[j].habitantes > 50000) \wedge \\
(\forall k: \ent)(0 \leq k < |ciudades| \implicaLuego ciudades[i].habitantes \geq 0) \wedge \\
(\forall m: \ent)(\forall n: \ent)(0 \leq m < n < |ciudades| \implicaLuego ciudades[m].nombre \neq ciudades[n].nombre)
}
\asegura{res = \sum_{i=0}^{|ciudades|-1} ciudades[i].habitantes}    
\end{proc}


$ \\

S_{1}: res := 0 

S_{2}: i := 0 

while \; (i  \textless \; ciudades.length) \; do  \par \par
         S_{3}: res := res + ciudades[i].habitantes \par
     
         S_{4}: i := i + 1 
     
endwhile    $
\\ \\

$P \equiv A \wedge B \wedge C$

$A \equiv (\exists l: \ent)(0 \leq l < |ciudades| \yLuego ciudades[l].habitantes > 50000)$ \par
$B \equiv (\forall k: \ent)(0 \leq k < |ciudades| \implicaLuego ciudades[k].habitantes \geq 0)$ \par
$C \equiv (\forall m: \ent)(\forall n: \ent)(0 \leq m < n < |ciudades| \implicaLuego ciudades[m].nombre \neq ciudades[n].nombre)$
\\ \par
$Pc \equiv res = 0 \; \wedge i = 0 \; \wedge A \wedge B \wedge C$
\\ \par

$B \equiv i < |ciudades|$
\\ \par

$I \equiv 0 \leq i \leq |ciudades| \yLuego res = \sum_{j=0}^{i-1} ciudades[j].habitantes$
\\ \par

$f_{v} \equiv |ciudades| - i$
\\ \par

Como la última instrucción del programa es el ciclo, podemos asumir que \par
\vspace{5px}

$Q \equiv Qc \equiv res = \sum_{j=0}^{|ciudades|-1} ciudades[j].habitantes$
\\

Por corolario de la propiedad de monotonía de las WP, para demostrar la correctitud del programa basta ver que son válidas las siguientes triplas de Hoare: \par
\vspace{5px}

\begin{center}
   ${ \{Pc\} S_{c} \{Qc \equiv{Q}\} \; y \; \{P\} S_{1};S_{2} \{Pc\}}$
\vspace{5px} 
\end{center}


\subsubsection{$Pc \implies wp(S_{3};S_{4},Qc)$ (Teorema del Invariante y Terminación)}

\paragraph{$Pc \implies I$}
$ Pc \equiv {res = 0 \; \wedge i = 0 \; \wedge A \wedge B \wedge C} \par$
\par Reemplazando en I y asumiendo el antecedente cómo verdadero:
\vspace{5px}
\begin{center}
$0 \leq 0  \leq |ciudades| \yLuego 0 = \sum_{j=0}^{-1} ciudades[j].habitantes $ \equiv I \par 
\vspace{5px}
$ 0 \leq |ciudades| \yLuego 0 = 0 \equiv I$ \\
\vspace{5px}
$ True \; \yLuego \; True \equiv I$ \\
\vspace{5px}
$True \equiv I$  \\
\vspace{5px}
$Pc \implies I\par$
\end{center}

\paragraph{$I \wedge B \implies wp(S_{3};S_{4}, I)$}
$wp(S_{3};S_{4}, I) \equiv wp(res = res + ciudades[i].habitantes, \; wp(i = i + 1, I))$ \equiv \par
\vspace{5px}
$wp(res = res + ciudades[i].habitantes, \; def(i+1) \yLuego I_{i+1}^{i})$ \equiv \par
\vspace{5px}
$wp(res = res + ciudades[i].habitantes, \; True \yLuego 0 \leq i + 1 \leq |ciudades| \yLuego res = \sum_{j=0}^{i}{ciudades[j].habitantes})$ \equiv \par
\vspace{5px}
$wp(res = res + ciudades[i].habitantes, \; \underbrace{-1 \leq i \leq |ciudades| - 1 \yLuego res = \sum_{j=0}^{i}{ciudades[j].habitantes}}_{I^{'}})$ \equiv \par
\vspace{5px}
$def(res + ciudades[i].habitantes) \yLuego (I^{'})_{res + ciudades.[i].habitantes}^{res}$ \equiv \par
\vspace{5px}
$0 \leq i < |ciudades| \; \yLuego $ -1 \leq i \leq |ciudades| - 1 \yLuego res + ciudades[i].habitantes = \sum_{j=0}^{i}{ciudades[j].habitantes}$ \equiv \par
\vspace{5px}
$0 \leq i < |ciudades| \yLuego res = \sum_{j=0}^{i-1}{ciudades[j].habitantes}$ \par
\vspace{20px}

$I \wedge B \equiv 0 \leq i \leq |ciudades| \yLuego res = \sum_{j=0}^{i-1} ciudades[j].habitantes \wedge i < |ciudades| \equiv$\par

\begin{center}
$0 \leq i < |ciudades| \yLuego res = \sum_{j=0}^{i-1} ciudades[j].habitantes  \equiv wp(S_{3};S_{4}, I)  $ \par
\vspace{5px}
\end{center}
\par Cómo asumimos el antecedente como verdadero:
\vspace{5px}
\begin{center}
$True \equiv wp(S_{3};S_{4}, I) \par
\vspace{5px}
    $I \wedge B \implies wp(S_{3};S_{4}, I)$
\end{center}

\paragraph{$I \wedge \neg B \implies Qc$}
$I \wedge \neg B \equiv 0 \leq i \leq |ciudades| \yLuego res = \sum_{j=0}^{i-1} ciudades[j].habitantes \wedge i \geq |ciudades| $ \equiv \par

\begin{center}
  $i = |ciudades| \yLuego res = \sum_{j=0}^{i-1} ciudades[j].habitantes$ \implies \par
\vspace{5px}  
$res = \sum_{j=0}^{|ciudades|-1} ciudades[j].habitantes \equiv Qc$  \par
\vspace{5px}
\end{center}
\par Cómo asumimos el antecedente como verdadero:
\vspace{5px}
\begin{center}
$True \equiv Qc $\par
\vspace{5px}
$I \wedge \neg B \implies Qc$
\end{center}

\paragraph{$I \wedge  B \wedge v_{0}=f_{v} \implies wp(S_{3};S_{4}, f_{v} < v_{0})$}

$wp(S_{3};S_{4}, f_{v} < v_{0}) \equiv wp(res = res + ciudades[i].habitantes, wp(i = i + 1, |ciudades| - i < v_{0}))$ \equiv \par
\vspace{5px}
$wp(res = res + ciudades[i].habitantes, def(i+1) \yLuego (|ciudades| - i < v_{0})_{i+1}^{i}) $ \equiv \par
\vspace{5px}
$wp(res = res + ciudades[i].habitantes, True \yLuego |ciudades| - i - 1 < v_{0})$ \equiv \par
\vspace{5px}
$def(res + ciudades[i].habitantes) \yLuego (|ciudades| - i - 1 < v_{0})_{res + ciudades[i].habitantes}^{res}$ \equiv \par
\vspace{5px}
$0 \leq i < |ciudades| \yLuego |ciudades| - i - 1 < v_{0}$  \par
\vspace{20px}
$I \wedge  B \wedge v_{0}=f_{v} \equiv 0 \leq i \leq |ciudades| \yLuego res = \sum_{j=0}^{i-1} ciudades[j].habitantes \wedge i < |ciudades| \wedge v_{0}=|ciudades| - i$ \equiv \par
\begin{center}
    $0 \leq i < |ciudades| \yLuego res = \sum_{j=0}^{i-1} ciudades[j].habitantes \wedge v_{0}=|ciudades| - i$
    \vspace{5px}
    \end{center}
   \par Reemplazando en $wp(S_{3};S_{4}, f_{v} < v_{0})$ y asumiendo el antecedente cómo verdadero:
    \vspace{5px}
    \begin{center}
    \vspace{5px}
    $0 \leq i < |ciudades| \yLuego |ciudades| - i - 1 < |ciudades| - i$ \equiv wp(S_{3};S_{4}, f_{v} < v_{0}) \par
    \vspace{5px}
    $True \yLuego True$ \equiv wp(S_{3};S_{4}, f_{v} < v_{0}) \par
    \vspace{5px}
    $True$ \equiv wp(S_{3};S_{4}, f_{v} < v_{0}) \par
    \vspace{5px}
    $I \wedge  B \wedge v_{0}=f_{v} \implies wp(S_{3};S_{4}, f_{v} < v_{0})$
    \end{center}



\paragraph{$I \wedge f_{v} \leq 0 \implies \neg B$}
$I \wedge f_{v} \leq 0 \equiv 0 \leq i \leq |ciudades| \yLuego res = \sum_{j=0}^{i-1} ciudades[j].habitantes \wedge |ciudades| - i \leq 0$ \equiv \par
\begin{center}
    $0 \leq i \leq |ciudades| \yLuego res = \sum_{j=0}^{i-1} ciudades[j].habitantes \wedge |ciudades| \leq i$ \equiv \par
    \vspace{5px}
    $i = |ciudades| \yLuego res = \sum_{j=0}^{i-1} ciudades[j].habitantes $ \implies \par
    \vspace{5px}
    $i \geq |ciudades|$ \equiv \neg B \par
    \vspace{5px}
\end{center}
\par Cómo asumimos el antecedente como verdadero:
\vspace{5px}
\begin{center}
    $True \equiv \neg B$ \par
    \vspace{5px}
    $I \wedge f_{v} \leq 0 \implies \neg B$
\end{center}

\subsubsection{$P \implies wp(S_{1};S_{2}, Pc)$}
$wp(S_{1};S_{2}, Pc) \equiv wp(res = 0, wp(i = 0, Pc)) \equiv wp(res = 0, def(0) \yLuego Pc_{0}^{i})$ \equiv \par
\vspace{5px}
$wp(res = 0, True \yLuego res = 0 \wedge 0 = 0 \wedge A \wedge B \wedge C) $ \equiv \par
\vspace{5px}
$def(0) \yLuego (res = 0 \wedge True \wedge A \wedge B \wedge C)_{0}^{res} $ \equiv \par
\vspace{5px}
$True \yLuego 0 = 0 \wedge A \wedge B \wedge C \equiv True \wedge A \wedge B \wedge C \equiv  A \wedge B \wedge C $ \ \par
\vspace{20px}
$P \equiv A \wedge B \wedge C \equiv  wp(S_{1};S_{2}, Pc)$
\vspace{5px}
\par Cómo asumimos el antecedente como verdadero:
\vspace{5px}
\begin{center}
$True \wedge True \wedge True \equiv wp(S_{1};S_{2}, Pc)$ \par
\vspace{5px}
$True \equiv wp(S_{1};S_{2}, Pc)$ \par
\vspace{5px}
    $P \implies wp(S_{1};S_{2}, Pc)$
\end{center}
\par El programa es correcto con respecto a su especificación.

\subsection{Demostrar que el valor devuelto es mayor a 50.000.}
Informalmente, a partir de la precondición del programa se puede asumir que existe algún elemento de la secuencia de entrada con más de 50.000 habitantes. Como el ciclo, mediante el paso de las iteraciones, suma los habitantes de cada ciudad de la entrada, no hay ciudades con un numero negativo de habitantes y el ciclo finaliza, se puede concluir que la suma total de habitantes dará como resultado al menos 50.000, cumpliendo así lo pedido. \par
Formalmente, para demostrar que el valor devuelto es mayor a 50.000, consideramos la siguiente postcondición Q y vemos si el programa sigue siendo correcto con respecto a su especificación.

Tomando $P, B, Pc, f_{v}$ igual que en el punto anterior, y \\ \par
$Q \equiv Qc \equiv res = \sum_{j=0}^{|ciudades|-1} ciudades[j].habitantes \wedge res > 50000$ \\ \par
$A \equiv (\exists l: \ent)(0 \leq l < |ciudades| \yLuego ciudades[l].habitantes > 50000)$\\ \par
$B \equiv (\forall k: \ent)(0 \leq k < |ciudades| \implicaLuego ciudades[k].habitantes \geq 0)$\\ \par
$I \equiv 0 \leq i \leq |ciudades| \yLuego res = \sum_{j=0}^{i-1} ciudades[j].habitantes \wedge A \wedge B$\\ \par
Para demostrar la correctitud del programa basta ver que son válidas las siguientes triplas de Hoare: \par
\vspace{5px}

\begin{center}
   ${ \{Pc\} S_{c} \{Qc \equiv{Q}\} \; y \; \{P\} S_{1};S_{2} \{Pc\}}$
\vspace{5px} 
\end{center}


\subsubsection{$Pc \implies wp(S_{3};S_{4},Qc)$ (Teorema del Invariante y Terminación)}

\paragraph{$Pc \implies I$}
$ Pc \equiv {res = 0 \; \wedge i = 0 \; \wedge A \wedge B \wedge C}
\vspace{5px}
\par Reemplazando en I y asumiendo el antecedente cómo verdadero:
\vspace{5px}
\begin{center}
$0 \leq 0  \leq |ciudades| \yLuego 0 = \sum_{j=0}^{-1} ciudades[j].habitantes \wedge A \wedge B $ \equiv I \par 
\vspace{5px}
$ 0 \leq |ciudades| \yLuego 0 = 0 \wedge A \wedge B \equiv I$ \\
\vspace{5px}
$ True \; \yLuego \; True \wedge True \wedge True \equiv I$ \\
\vspace{5px}
$True \equiv I$  \\
\vspace{5px}
$Pc \implies I\par$
\end{center}


\paragraph{$I \wedge B \implies wp(S_{3};S_{4}, I)$}
$wp(S_{3};S_{4}, I) \equiv wp(res = res + ciudades[i].habitantes, \; wp(i = i + 1, I))$ \equiv \par
\vspace{5px}
$wp(res = res + ciudades[i].habitantes, \; def(i+1) \yLuego I_{i+1}^{i})$ \equiv \par
\vspace{5px}
$wp(res = res + ciudades[i].habitantes, \; True \yLuego 0 \leq i + 1 \leq |ciudades| \yLuego res = \sum_{j=0}^{i}{ciudades[j].habitantes} \wedge A \wedge B)$ \equiv \par
\vspace{5px}
$wp(res = res + ciudades[i].habitantes, \; \underbrace{-1 \leq i \leq |ciudades| - 1 \yLuego res = \sum_{j=0}^{i}{ciudades[j].habitantes} \wedge A \wedge B}_{I^{'}})$ \equiv \par
\vspace{5px}
$def(res + ciudades[i].habitantes) \yLuego (I^{'})_{res + ciudades.[i].habitantes}^{res}$ \equiv \par
\vspace{5px}
$0 \leq i < |ciudades| \; \yLuego -1 \leq i \leq |ciudades| - 1 \yLuego res + ciudades[i].habitantes = \sum_{j=0}^{i}{ciudades[j].habitantes} \wedge A \wedge B$ \equiv \par
\vspace{5px}
$0 \leq i < |ciudades| \yLuego res = \sum_{j=0}^{i-1}{ciudades[j].habitantes} \wedge A \wedge B$ \par
\vspace{20px}


$I \wedge B \equiv 0 \leq i \leq |ciudades| \yLuego res = \sum_{j=0}^{i-1} ciudades[j].habitantes \wedge A \wedge B \wedge i < |ciudades| \equiv$\par

\begin{center}
$0 \leq i < |ciudades| \yLuego res = \sum_{j=0}^{i-1} ciudades[j].habitantes  \wedge A \wedge B \equiv wp(S_{3};S_{4}, I)$ \par
\vspace{5px}
\end{center}
\par Cómo asumimos el antecedente como verdadero:
\vspace{5px}
\begin{center}
$True \equiv wp(S_{3};S_{4}, I) \par
\vspace{5px}
    $I \wedge B \implies wp(S_{3};S_{4}, I)$
\end{center}

\paragraph{$I \wedge \neg B \implies Qc$}
$I \wedge \neg B \equiv 0 \leq i \leq |ciudades| \yLuego res = \sum_{j=0}^{i-1} ciudades[j].habitantes \wedge A \wedge B \wedge i \geq |ciudades| $ \equiv \par

\begin{center}
  $i = |ciudades| \yLuego res = \sum_{j=0}^{i-1} ciudades[j].habitantes \wedge A \wedge B$ \implies \par
\vspace{5px}  
$res = \sum_{j=0}^{|ciudades|-1} ciudades[j].habitantes \wedge A \wedge B $  \par
\vspace{5px}
\end{center}
\par Cómo sabemos que A nos asegura que haya al menos un elemento de ciudades mayor a 50000, B nos asegura que todos los elementos de ciudades son no negativos y asumimos el antecedente como verdadero, se tiene que:
\vspace{5px}
\begin{center}
$res = \sum_{j=0}^{|ciudades|-1} ciudades[j].habitantes \wedge A \wedge B \implies$ \par 
\vspace{5px}
$res = \sum_{j=0}^{|ciudades|-1} ciudades[j].habitantes \wedge res > 50000 \equiv Qc$ \par
\vspace{5px}
$True \wedge True \equiv Qc$ \par
\vspace{5px}
$True \equiv Qc$ \par
\vspace{5px}
$I \wedge \neg B \implies Qc$
\end{center}

\paragraph{$I \wedge  B \wedge v_{0}=f_{v} \implies wp(S_{3};S_{4}, f_{v} < v_{0})$}

$wp(S_{3};S_{4}, f_{v} < v_{0}) \equiv wp(res = res + ciudades[i].habitantes, wp(i = i + 1, |ciudades| - i < v_{0}))$ \equiv \par
\vspace{5px}
$wp(res = res + ciudades[i].habitantes, def(i+1) \yLuego (|ciudades| - i < v_{0})_{i+1}^{i}) $ \equiv \par
\vspace{5px}
$wp(res = res + ciudades[i].habitantes, True \yLuego |ciudades| - i - 1 < v_{0})$ \equiv \par
\vspace{5px}
$def(res + ciudades[i].habitantes) \yLuego (|ciudades| - i - 1 < v_{0})_{res + ciudades[i].habitantes}^{res}$ \equiv \par
\vspace{5px}
$0 \leq i < |ciudades| \yLuego |ciudades| - i - 1 < v_{0}$ \equiv \par
\vspace{20px}
$I \wedge  B \wedge v_{0}=f_{v} \equiv 0 \leq i \leq |ciudades| \yLuego res = \sum_{j=0}^{i-1} ciudades[j].habitantes \wedge A \wedge B \wedge i < |ciudades| \wedge v_{0}=|ciudades| - i$ \par
\begin{center}
    $\equiv 0 \leq i < |ciudades| \yLuego res = \sum_{j=0}^{i-1} ciudades[j].habitantes \wedge A \wedge B \wedge v_{0}=|ciudades| - i$
    \vspace{5px}
    \end{center}
   \par Reemplazando en $wp(S_{3};S_{4}, f_{v} < v_{0})$ y asumiendo el antecedente cómo verdadero:
    \vspace{5px}
    \begin{center}
    \vspace{5px}
    $0 \leq i < |ciudades| \yLuego |ciudades| - i - 1 < |ciudades| - i$ \equiv wp(S_{3};S_{4}, f_{v} < v_{0}) \par
    \vspace{5px}
    $True \yLuego True$ \equiv wp(S_{3};S_{4}, f_{v} < v_{0}) \par
    \vspace{5px}
    $True$ \equiv wp(S_{3};S_{4}, f_{v} < v_{0}) \par
    \vspace{5px}
    $I \wedge  B \wedge v_{0}=f_{v} \implies wp(S_{3};S_{4}, f_{v} < v_{0})$
    \end{center}




\paragraph{$I \wedge f_{v} \leq 0 \implies \neg B$}
$I \wedge f_{v} \leq 0 \equiv 0 \leq i \leq |ciudades| \yLuego res = \sum_{j=0}^{i-1} ciudades[j].habitantes \wedge A \wedge B \wedge |ciudades| - i \leq 0$ \equiv \par
\begin{center}
    $0 \leq i \leq |ciudades| \yLuego res = \sum_{j=0}^{i-1} ciudades[j].habitantes \wedge A \wedge B \wedge |ciudades| \leq i$ \equiv \par
    \vspace{5px}
    $i = |ciudades| \yLuego res = \sum_{j=0}^{i-1} ciudades[j].habitantes \wedge A \wedge B$ \implies \par
    \vspace{5px}
    $i \geq |ciudades|$ \equiv \neg B \par
    \vspace{5px}
\end{center}
\par Cómo asumimos el antecedente como verdadero:
\vspace{5px}
\begin{center}
    $True \equiv \neg B$ \par
    \vspace{5px}
    $I \wedge f_{v} \leq 0 \implies \neg B$
\end{center}

\subsubsection{$P \implies wp(S_{1};S_{2}, Pc)$}
$wp(S_{1};S_{2}, Pc) \equiv wp(res = 0, wp(i = 0, Pc)) \equiv wp(res = 0, def(0) \yLuego Pc_{0}^{i})$ \equiv \par
\vspace{5px}
$wp(res = 0, True \yLuego res = 0 \wedge 0 = 0 \wedge A \wedge B \wedge C) $ \equiv \par
\vspace{5px}
$def(0) \yLuego (res = 0 \wedge True \wedge A \wedge B \wedge C)_{0}^{res} $ \equiv \par
\vspace{5px}
$True \yLuego 0 = 0 \wedge A \wedge B \wedge C \equiv True \wedge A \wedge B \wedge C \equiv  A \wedge B \wedge C $ \ \par
\vspace{20px}
$P \equiv A \wedge B \wedge C \equiv  wp(S_{1};S_{2}, Pc)$
\vspace{5px}
\par Cómo asumimos el antecedente como verdadero:
\vspace{5px}
\begin{center}
$True \wedge True \wedge True \equiv wp(S_{1};S_{2}, Pc)$ \par
\vspace{5px}
$True \equiv wp(S_{1};S_{2}, Pc)$ \par
\vspace{5px}
    $P \implies wp(S_{1};S_{2}, Pc)$
\end{center}
\par El programa es correcto con respecto a su especificación y devuelve un resultado mayor a 50.000.

\end{document}
